\documentclass[12pt,a4paper]{article}

% incluyendo paquetes
\usepackage[utf8]{inputenc}
\usepackage[spanish]{babel}
\usepackage{milibreria}
\usepackage{animate}



\graphicspath{{C:/Users/HUAWEI/Pictures/imagesppt/}} %\incluye todos las imágenes de esa ruta
%\graphicspath{{D:/proyectos_latex/7mo_semestre/gestion_redes/informe_de_redes/main/images}}
\begin{document} % inicio  de documento 

\include{caratula.tex} % incluyendo la caratula
\tableofcontents % índice automático
\pagestyle{fancy} \mystyle \newpage % Aplicar el estilo de encabezado y pie de página
% inicio del documento  
\newcounter{step}
\newcommand{\dpsec}{Dirección de Proyección Social y Extensión Cultural}


\section{I. Planificacion de la Auditoria}

\section{Información de la organización}
\subsection{Modelamiento De La Empresa}
\subsubsection{Nombre De La Organización}
\textbf{\dpsec}

\subsubsection{Descripción de la Empresa}
La Dirección de Proyección Social y Extensión Cultural, es un órgano dependiente del Vicerrectorado Académico, responsable de promover, organizar, dirigir y supervisar las actividades en el marco del desarrollo sostenible de Proyección Social y Extensión Cultural de la universidad, orientados hacia la comunidad, con el fin de coadyuvar al desarrollo de la región, revalorando la identidad cultural.

\subsubsection{Misión, Visión y Valores}
\subsection*{Mision}
La Dirección de Proyección Social y Extensión Cultural es generadora de la interacción con la comunidad para el desarrollo regional a través de la ciencia, la tecnología y las expresiones culturales como parte de la labor académica y de investigación en el marco de la proyección social y extensión cultural sostenible.

\subsubsection*{Vision}
La Dirección de Proyección Social y Extensión Cultural fortalece el desarrollo de la región, bajo los principios de la ciencia, la tecnología, los valores culturales y la investigación en el marco de una responsabilidad social sostenible.

\subsection{Servicios y registro institucional}
%\subsubsection{Información Escale (Organización)}
\subsubsection{Horarios de Atencion, Contactos, Portar WEB}
\begin{table}[!hbt]
    \centering
    \begin{tabular}{cc}
    \toprule
    \textbf{Nombre } & \textbf{Información} \\ 
    \midrule
    Dirección: & Auditorio Magno UNAP, Puno 21001 \\
    Directora & DRA. MILDER ZANABRIA ORTEGA \\
    Correo electrónico: & drs unap.edu.pe \\
    Jefe de PSEU & M. SC. WILKERSON \\
    Jefe de SDG & ING. YUMY ROMERO  \\
    Jefa de GA & BLGO.ANGEL CANALES \\
    
    Facebook: & \href{https://www.facebook.com/p/Direcci%C3%B3n-de-Proyecci%C3%B3n-Social-y-Extensi%C3%B3n-Cultural-UNA-Puno-100071137256988/}{Ver enlac} \\
    Pagina web & \href{https://proyeccionsocial.com/}{https://proyeccionsocial.com/}\\
    Horario de atención: & lunes a viernes de 8:30 a 16:30 \\
    Celular & 950 036 674\\
    Distrito:& Puno \\
    Provincia:& Puno \\
    Región:& Puno \\

\bottomrule
\end{tabular}
\caption{Tabla de Datos de la organizacion}
\label{tabla:ejemplo}
\end{table}

\newpage
\subsubsection*{Frontis de la DPSEC}
\begin{figure}[!htb]
    \centering
    \animategraphics[autoplay,loop,width=0.4\textwidth]{2}{images/frontis}{1}{2}
    \caption{Frontis de la oficina de la \dpsec}
\end{figure}

\newpage
\subsubsection*{Ubicacion por vista satelital}
Perú, Puno, Puno, 5XGM+897, Puno 21001 (Auditorio Magno UNAP, Puno 21001)
\\
Latitud -15.824222019301729 \\
Longitud -70.01674037356099



\begin{figure}[!htb]
    \centering
    \includegraphics[width=0.7\textwidth]{images/ubicacion.png}
    \caption{Ubicacion vista MAPS } \par \textit{Imagen descargada de google maps}

\end{figure}


\newpage
\section{II. INTRODUCCIÓN DE LA AUDITORÍA}
\subsection*{Dirección de Proyección Social y Extensión Cultural (DPSEC)}

La organización a la cual se realizará la auditoría es un órgano dependiente del Vicerrectorado Académico de la universidad. Ubicada en el campus universitario principal, es responsable de promover, organizar, dirigir y supervisar las actividades en el marco del desarrollo sostenible de Proyección Social y Extensión Cultural, orientadas hacia la comunidad, con el fin de coadyuvar al desarrollo de la región, revalorando la identidad cultural.
\espacio
La DPSEC cuenta con su propio reglamento interno, que establece los derechos y deberes de los participantes y colaboradores. Su infraestructura moderna incluye todos los servicios básicos, oficinas administrativas, salas de reuniones, auditorios para eventos culturales, y áreas destinadas a la interacción comunitaria.

\newpage
\section{III. OBJETIVOS DE LA AUDITORÍA}
\subsection{Definición de objetivo}
Un objetivo es una meta o propósito específico que una persona u organización se propone alcanzar en un plazo determinado.
\\
Metas: Fines específicos de la auditoría.
\\
Plazo: Los términos en unidades de tiempo en que se satisface el fin que se
pretende con la auditoría.


\subsection*{Objetivo general}
Evaluar y asegurar la efectividad de las políticas, procedimientos y prácticas de seguridad de la Dirección de Proyección Social y Extensión Cultural (DPSEC) para garantizar la protección de los recursos humanos, físicos y de información, contribuyendo al cumplimiento de sus objetivos institucionales y al desarrollo sostenible de la Universidad asi mismo recomendar algunas soluciones que beneficie a la organizacion.

\subsection*{Objetivos específicos}
\subsubsection*{Revisar Políticas y Procedimientos de Seguridad} Evaluar la existencia, adecuación y cumplimiento de las políticas y procedimientos de seguridad establecidos por la DPSEC.

\subsubsection*{Identificar Riesgos y Vulnerabilidades} Detectar posibles riesgos y vulnerabilidades en los sistemas de seguridad física, de la información y de las operaciones diarias.

\subsubsection*{Evaluar la Efectividad de los Controles de Seguridad} Analizar la efectividad de los controles de seguridad implementados para proteger los activos físicos y digitales de la organización.

\subsubsection*{Revisar la Gestión de Incidentes de Seguridad} Evaluar los procedimientos de gestión y respuesta ante incidentes de seguridad, así como la capacidad de la DPSEC para reaccionar y recuperarse de eventos adversos.

\subsubsection*{Verificar la Capacitación y Concientización en Seguridad} Comprobar que el personal de la DPSEC recibe la capacitación adecuada y está concientizado sobre las políticas y procedimientos de seguridad.

\subsubsection*{Recomendar Mejoras y Acciones Correctivas} Proponer mejoras y acciones correctivas para fortalecer el sistema de seguridad de la DPSEC, basadas en los hallazgos de la auditoría.

\subsubsection*{Evaluar el Cumplimiento Normativo} Verificar que la DPSEC cumpla con las normativas y regulaciones aplicables en materia de seguridad.

\newpage
\section{IV. Visita Preliminar}
\subsection{¿Cómo se encuentran distribuidos los sistemas en el área?}
Abarca toda el area de la oficina teniendo el area de trabajos, area de redes, area de recepcion y el area de la direccion.
\begin{figure}[!htb]
    \centering
    \includegraphics[width=0.7\textwidth]{images/AreaTrabajo.jpeg}
    \caption{Imagen al interior de la dpsec}

\end{figure}
\subsection*{¿Cuántos, cuáles, cómo y de qué tipo son los equipos que están instalados
en el centro de sistemas?}
7 en funcionamiento dentro del area de trabajo de un total de 8 (en total son 6 equipos de computo)
\begin{figure}[!htb]
    \centering
    \animategraphics[autoplay,loop,width=0.4\textwidth]{0.5}{images/equipo}{1}{4}
    \caption{Equipos de la depsec en total}
\end{figure}

\begin{figure}[!htb]
    \centering
    \animategraphics[autoplay,loop,width=0.4\textwidth]{0.5}{images/datos}{1}{3}
    \caption{Información de la información que maneja}
\end{figure}

\subsection*{¿Qué tipo de instalaciones y conexiones físicas hay en el área de sistemas, y
cómo están distribuidas?}
Cableado  para la fuente de energia, tambien para los puntos de acceso a internet, cableado en el techo y los cableados para las computadoras.

\begin{figure}[!htb]
    \centering
    \animategraphics[autoplay,loop,width=0.4\textwidth]{0.5}{images/fisica}{1}{4}
    \caption{instalaciones fisicas de cable}
\end{figure}

\newpage
\subsubsection*{Contacto Inicial}
Observe su reaccion al llevar el documenot de solicitud para realizar la auditoria un miedo a que no me proporcionaran dicha informaciónq 
que yo nesecitaria sim embargo la respuesta fue algo similar pero que resultados positivos me acepto pero nos dijeron que nosotros tenemos que presentar un documento 
especialmente como ellos quieren para fortalecer los puntos debiles encontrada por nosotros los auditores.
\begin{figure}[!htb]
    \centering
    \includegraphics[width=0.4\textwidth]{images/solicitud.jpeg}
    \caption{foto de evidencia Solicitud }

\end{figure}

\subsection*{¿Cómo reacciona el personal ante la visita del auditor? ¿Cuáles son las
medidas de seguridad visibles que existen?}
El personal encargado de nuestra visita era una egresada de la carrera de ingeniera de sistemas
nos ayudo en muchas cosas nos explico y corrigio en algunos aspectos cabe recalcar que el personal era
un egresado de mucho tiempo y se mostro una actitud de querer volver a aprender sobre la auditoria.
Pero nos mostro cada area cada peticion que le pedimos nos dijo tal cual.

\newpage
\section{V. ORGANIGRAMA DE LA ORGANIZACIÓN}
\begin{figure}[!htb]
    \centering
    \includegraphics[width=0.6\textwidth]{images/organigrama.jpeg}
    \caption{organigrama de la organización}

\end{figure}

\newpage
\section{VI. ÁREAS A SER AUDITADAS DE LA ORGANIZACIÓN}
El area principa donde esta la oficina principal solo tiene un area, asi que específicamente auditare en la Dirección de Proyección Social y Extensión Cultural (DPSEC) en términos de seguridad son las siguientes:
\begin{figure}[!htb]
    \centering
    \includegraphics[width=0.6\textwidth]{images/AreaTrabajo.jpeg}
    \caption{Area donde esta todo en un ambiente. Como al fondo podemos ver el area de redes}
    
\end{figure}
\begin{itemize}
    \item \textbf{Políticas y Procedimientos de Seguridad:} Revisión y evaluación de las políticas y procedimientos de seguridad existentes, asegurando que estén documentados, actualizados y se cumplan adecuadamente.
    \item \textbf{Gestión de Riesgos y Vulnerabilidades:} Identificación y análisis de los riesgos y vulnerabilidades que pueden afectar la seguridad física, de la información y operativa de la DPSEC.
    \item \textbf{Controles de Seguridad:} Evaluación de la efectividad de los controles de seguridad implementados para proteger los activos físicos y digitales, incluyendo acceso físico a instalaciones y medidas de ciberseguridad.
    \item \textbf{Gestión de Incidentes de Seguridad:} Revisión de los procedimientos y protocolos para la gestión y respuesta ante incidentes de seguridad, así como la capacidad de recuperación ante eventos adversos.
    \item \textbf{Capacitación y Concientización en Seguridad:} Verificación de los programas de capacitación y concientización en seguridad destinados al personal de la DPSEC, asegurando que todos los empleados conozcan y cumplan con las políticas y procedimientos de seguridad.
    \item \textbf{Cumplimiento Normativo:} Evaluación del cumplimiento de la DPSEC con las normativas y regulaciones aplicables en materia de seguridad.
    \item \textbf{Infraestructura y Equipamiento de Seguridad:} Inspección de la infraestructura y el equipamiento de seguridad disponibles, como cámaras de vigilancia, sistemas de alarma, controles de acceso, entre otros.
    
\end{itemize}

\newpage
\section{VII. TIPO DE AUDITORÍA A UTILIZAR}
\subsection{Auditoria de la seguridad de los sistemas computacionales}
Dada la estructura de la organización DPSEC, donde todos los equipos de cómputo y redes se encuentran centralizados en un único área, es crucial garantizar la integridad, disponibilidad y confidencialidad de los sistemas. 
Esta auditoría de seguridad permitirá identificar vulnerabilidades, evaluar riesgos y asegurar que se implementen las mejores prácticas de protección de la información y los activos tecnológicos. 
Además, ayudará a cumplir con las normativas y estándares de seguridad, asegurando la continuidad operativa y la confianza de la Universidad y partes interesadas.
\begin{figure}[!htb]
    \centering
    \includegraphics[width=0.6\textwidth]{images/AreaTrabajo.jpeg}
    \caption{Area de la dpsec}
    
\end{figure}

\newpage
\section{VIII. INSTRUMENTOS DE AUDITORÍA}
Señalar los instrumentos y herramientas de auditoría que utilizará y justificar la importancia que
este tendrá en la Auditoría.


\newpage
\cite{prueba}
\section{Referencias}
\bibliographystyle{apacite}
\bibliography{referencias.bib}


\end{document}